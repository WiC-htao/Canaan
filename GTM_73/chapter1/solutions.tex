\documentclass[a4paper]{article}

\usepackage{../../header}

\newcommand{\thechapter}{1}

\renewcommand{\theexercise}{\thechapter.\arabic{section}.\arabic{exercise}}

\renewcommand{\thetheorem}{\thechapter.\arabic{section}.\arabic{theorem}}

\title{GTM 73 Chapter{\thechapter} Solutions}
\author{htao}

\begin{document}
\maketitle
\begin{exercise}

\end{exercise}
\begin{solution}
    $\NN^*$ and $\NN$
\end{solution}
\begin{exercise}

\end{exercise}
\begin{solution}
    Nothing more than a easy verification.
\end{solution}

\begin{exercise}

\end{exercise}
\begin{solution}
    No. A counterexample is
    \[
        G=\Dset{
            \begin{pmatrix}
                a & b \\
                0 & 0 \\
            \end{pmatrix}
        }{a,b,c,d\in \RR}
    \]
    with left identity
    \[
        \begin{pmatrix}
            1 & 0 \\
            0 & 0
        \end{pmatrix}
    \]
    and right inverse
    \[
        \begin{pmatrix}
            a^{-1} & 0 \\
            0      & 0
        \end{pmatrix}
    \]
    However, it has other left identities
    \[
        \begin{pmatrix}
            1 & x \\
            0 & 0
        \end{pmatrix},\ \forall x\in \RR
    \]
    whence it is not a group.
\end{solution}
\end{document}