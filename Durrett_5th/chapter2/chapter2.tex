\documentclass[a4paper]{article}
\usepackage{amsmath}
\usepackage{amssymb}
\usepackage{../../math_symbol}
\usepackage{../../header}

\newcommand{\thechapter}{2}
\title{Durrett 5th Chapter{\thechapter} Solutions}
\author{htao}


\makeatletter
\def\thm@space@setup{\thm@preskip=0.3 \baselineskip}
\makeatother


\usepackage{physics}


\renewcommand{\theexercise}{\thechapter.\arabic{section}.\arabic{exercise}}

\renewcommand{\thetheorem}{\thechapter.\arabic{section}.\arabic{theorem}}


\begin{document}
\maketitle

\section{Independence}
\begin{exercise}
    Suppose $(X_1, \dots ,X_n)$ has density $f(x_1, x_2,\dots , x_n)$, that is
    \[
        P((X_1,X_2, \dots ,X_n) \in A) =\int_A f(x) \dif x \mtext{for} A\in \mathcal{R}^n
    \]
    If $f(x)$ can be written as $g_1(x_1)\cdots g_n(x_n)$ where the $g_m\ge 0$ are measurable,
    then $X1,X2, \dots ,Xn$ are independent. Note that the $g_m$ are not assumedto be probability densities.
\end{exercise}
\begin{solution}
    Let $A=B_1\times\dots\times B_n$ where $B_i\in \mathcal{B}$. Then by Fubini-Tonelli Theorem
    \begin{align*}
        P(X_1\in B_1, \dots, X_n\in B_n) & = P((X_1, \dots, X_n)\in A)                                                 \\
                                         & = \int_A f(x)\dif x = \int_{B_1\times\dots\times B_n} g_1\cdots g_n  \dif x \\
                                         & = \prod_{1\le i\le n} \int_{B_i}g_i\dif x
    \end{align*}
    Then let $A_i=\RR\times\dots \times\RR\times B_i \times\RR\times\dots\times\RR$, then since $\prod_{1\le j\le n} \int_{\RR}g_j\dif x = \int_{\RR^n} f\dif x = 1$,
    \[
        P(X_i\in B_i) = P((X_1, \dots, X_n)\in A_i) = \frac{\int_{B_i} g_i\dif x}{\int_{\RR} g_i\dif x}{\prod_{1\le j\le n} \int_{\RR}g_j\dif x}=\frac{\int_{B_i} g_i\dif x}{\int_{\RR} g_i\dif x}
    \]
    The desired result follows from
    \begin{align*}
        \prod_{1\le j\le n} P(X_i\in B_i) & = \frac{\prod_{1\le j\le n} \int_{B_i}g_j\dif x}{\prod_{1\le j\le n} \int_{\RR}g_j\dif x} \\
                                          & =\prod_{1\le j\le n} \int_{B_i}g_j\dif x =  P(X_1\in B_1, \dots, X_n\in B_n)
    \end{align*}
\end{solution}

\begin{exercise}
    Suppose $X_1,\dots ,X_n$ are random variables that take values in countable sets $S_1, \dots, S_n$. Then in order for $X_1, . . . ,X_n$ to be independent,
    it is sufficient that whenever $x_i \in S_i$,
    \[
        P(X_1=x_1, \dots, X_n=x_n)=\prod_{i=1}^n P(X_i=x_i)
    \]
\end{exercise}
\begin{solution}
    For any measurable set $B_i\subset S_i$, $B_i$ is countable and $B_1\times\dots\times B_n$ is countable.
    Then
    \begin{align*}
        P(X_1\in B_1, \dots, X_n\in B_n) & = \sum_{(x_1, \dots, x_n)\in B_1\times\dots\times B_n}  P(X_1=x_1, \dots, X_n=x_n) \\
                                         & =\sum_{(x_1, \dots, x_n)\in B_1\times\dots\times B_n}  \prod_{i=1}^n P(X_i=x_i)    \\
                                         & = \prod_{i=1}^n (\sum_{x_i\in B_i} P(X_i=x_i)) = \prod_{i=1}^n P(X_i\in B_i)
    \end{align*}
    is followed by the desired result.
\end{solution}


\section{}
\section{Borel-Cantelli Lemmas}
\begin{exercise}
    Prove that $P(\limsup A_n) \ge \limsup P(A_n)$ and $P(\liminf An) \le \liminf P(A_n)$.
\end{exercise}
\begin{proof}
    \[
        P(\limsup A_n) = P(\bigcap_{n=1}^\infty \bigcup_{m=n}^\infty A_m) = \lim_{n\to\infty} P(\bigcup_{m=n}^\infty A_m)\ge \lim_{n\to\infty} \sup_{m\ge n} P(A_m) = \limsup P(A_n)
    \]
    It can be proved by the similar method for $\liminf$ or directly use Fatou's Lemma to $1_{A_n}$

\end{proof}
\section{}
\section{}
\section{}
\section{}

\subsection*{exercise}
\begin{exercise}
    Consider $\gamma(a)$ defined in (2.7.1). The following are equivalent:
    \begin{enumerate}[label=(\alph*)]
        \item $\gamma(a)=-\infty$
        \item $P(X\ge a) =0$
        \item $P(S_n\ge na)=0$
    \end{enumerate}
\end{exercise}
\begin{solution}
    According the context,
    \[
        \gamma(a) := \lim_{n\to\infty} \frac{1}{n} \log P(S_n\ge na)
    \]
    where $Sn=X_1+\dots +X_n$ and $X_i$'s are i.i.d and $a> EX_i$. Its well-definition
    has been proven in DTE5th and is repeated in \cref{thm2.7.2}

    \begin{itemize}
        \item (a)$\Rightarrow$(b). Assuming $P(X_1\ge a)=r>0$, then we have the contradiction that
              \[
                  \gamma(a) = \lim_{n\to\infty} \frac{1}{n}\ln P(S_n\ge na) \ge \lim_{n\to\infty} \frac{1}{n}\ln \left(P(X_i\ge a)\right)^n=\ln r>-\infty
              \]
        \item (b)$\Rightarrow$(c) is trivial
        \item (c)$\Rightarrow$(a) is weird but trivial.
    \end{itemize}
\end{solution}


\appendix
\section{Related Theorem Details}

\stepcounter{section}
\stepcounter{section}
\stepcounter{section}
\stepcounter{section}
\stepcounter{section}

\subsection*{2.7}
\stepcounter{section}
\begin{lemma}[Lemma 2.7.1 in DTE5th]
    \label{lemma2.7.1}
    If $\gamma_{n+m}\ge\gamma_n+\gamma_m$ holds for any $m$ and n, then $\gamma(n)/n\to \sup_k \gamma(k)/k$ as $n\to\infty$.
\end{lemma}
\begin{proof}
    We have $\limsup \frac{\gamma_n}{n} \le \sup_k \frac{\gamma_k}{k}$ and let $n=km+l$, then
    \[
        \frac{\gamma_n}{n} \ge \frac{km}{n} \frac{\gamma_m}{m} + \frac{\gamma_l}{n}
    \]
    which conclude $\liminf \gamma_n/n \ge \gamma_m/m$ for any $m$ and hence the desired result.
\end{proof}

\begin{theorem}
    \label{thm2.7.2}
    Let  $X_i$'s are i.i.d and $Sn=X_1+\dots +X_n$. For any $a>EX$,
    \[
        \lim_{n\to\infty} \frac{1}{n} \log P(S_n\ge na)
    \]
    exists on $\RR\cup\Set{-\infty}$.
\end{theorem}
\begin{proof}
    We observe that
    \[
        \ln P(S_{n+m}\ge(n+m)a) > \ln P(Sn\ge na) + \ln P(S_m\ge ma)
    \]

    and hence by \cref{lemma2.7.1} as $n\to\infty$,
    \[
        \frac{\ln P(Sn\ge na) }{n} \to \sup_m \frac{ \ln P(S_m\ge ma)}{m} < 0
    \]

\end{proof}
\end{document}